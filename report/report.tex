\documentclass[twocolumn,letterpaper,10pt]{article}
%\usepackage[T1]{fontenc}
\usepackage{amsmath}
\usepackage{listings}
\usepackage{booktabs}
% The following is needed in order to make the code compatible
% with both latex/dvips and pdflatex.
\ifx\pdftexversion\undefined
\usepackage[dvips]{graphicx}
\else
\usepackage{graphicx}
\usepackage{algorithms}
\DeclareGraphicsRule{*}{mps}{*}{}
\fi

\newcommand{\naive}{na\"{\i}ve }


\title{CSEE4824 Course Project}
\author{Danqing Hua $<dh2604@columbia.edu>$\\
  Jiacheng Yang $<jy2522@columbia.edu>$}
\date{December 2011}

\begin{document}

\maketitle

\section{Introduction}

Saadaval on mitmeid erinevaid Lorem Ipsumi variante, kuid enamik neist
on l�bi teinud muutusi kas sisestatud huumori v�i suvaliste s�nade t�ttu,
mis ei n�e �ldsegi usutavad v�lja. Kui te kavatsete Lorem Ipsumit kasutada,
peate te kindel olema, et kuskil keset teksti midagi piinlikku ei ole. K�ik
Lorem Ipsumi generaatorid kipuvad kindlaid lauseridu kordama, seega on see
sait siin esimene t�eline Lorem Ipsumi generaator Internetis. See kasutab
s�nastikku �le 200 ladina s�naga ja, koos hulga lausestruktuuride mudelitega,
toodab korralikku Lorem Ipsumit. Toodetud Lorem Ipsum on seega alati vaba
kordustest, naljadest, ebaharilikest s�nadest jne.~\cite{clrs}.

\subsection{Something}

asdlfjklasdfjlkj. See Table~\ref{tab:foo}.

\begin{table}[ht!]
\begin{center}
\begin{tabular}{lllrr}
\toprule
 Alg.     &  L1d miss  &  L2 miss  & \# instr  &  time ($ms$)  \\
\midrule
 a      &  0.09\%    &  96.86\%  &  94.35e5  &    516.16  \\
 b      &  0.12\%    &  96.94\%  &  74.38e5  &    423.77  \\
 c      &  0.66\%    &  96.86\%  &  15.39e7  &   84.34e2  \\
 d      &  0.20\%    &  92.61\%  &  11.54e7  &   66.07e2  \\
 e      &  1.49\%    &  99.50\%  &  83.98e7  &  46.23e3  \\
 f      &  0.21\%    &  98.02\%  &  62.55e7  &  35.84e3  \\
\bottomrule
\end{tabular}
\end{center}
\caption{Sample table.\label{tab:foo}}
\end{table}

\subsection{Or Another}

Vastupidiselt tavaarusaamale, Lorem Ipsum ei ole lihtsalt suvaline tekst.
~(Figure~\ref{fig:alg}.)

\begin{figure}[htb]
\centering
\begin{lstlisting}
while (1) {
  a = a \% b;
  if (a == 0)
    return b;
}
\end{lstlisting}
\caption{\label{fig:alg}Some code snippet.}
\end{figure}

\section{Software Optimization}

Among the four operations, namely scaling, addition, vector
multiplication and matrix multiplication, we focus our optimization to
matrix multiplication, because it has the highest computational
complexity $O(n^3)$. Amdahl's law suggests us that we need to improve
the dominant operation to achieve significant speed-up.

So our following discussion will focus on the matrix
multiplication. We will quickly review the techniques we use to
optimizing the other three operations as well at the end of this section.

\subsection{Loop Ordering}
First of all, for matrix multiplication, there are mutliple ways to order the
loops. We choose the i,j,k ordering in our implementation, because
we are able to use a register to store the temporary result
$r(i,j)$. Other ordering may enable us to reference elements in input
matrix A or B by register. However, in general writing memory is less
efficient than reading from memory, so it's better for us save the
writing to register. We also benefit from spatial locality in both
matrices after we transpose matrix $B$.

\subsection{Transpose Matrix B}

\begin{algorithmic}
  \FOR {$i = 1 \to size$}
  \FOR {$j = 1 to size$}
  \STATE $t \gets 0$
  \FOR {$k = 1 to size$}
      \STATE $t = t + a[i][k] * b[k][j]$
  \ENDFOR
  \STATE $r[i][j] = t$
  \ENDFOR
  \ENDFOR
\end{algorithmic}

\subsection{Register Reusing}



\subsection{Loop Unrolling}
Since we are not allowed to use any compiler optimization, we have to
do loop unrolling ourselves. Loop unrolling has at least the following
two advantages: (1) reduce branches (2) use registers to avoid memory
access.

\subsection{Cache Blocking}
\subsection{Multi-threading}

\subsection{Optimizing other operations}

\section{Conclusion}

Lorem Ipsum on lihtsalt proovitekst, mida kasutatakse printimis- ja
ladumist��stuses. See on olnud t��stuse p�hiline proovitekst juba alates
1500. aastatest, mil tundmatu printija v�ttis hulga suvalist teksti, et
teha tr�kin�idist. Lorem Ipsum ei ole ainult viis sajandit s�ilinud, vaid on
ka edasi kandunud elektroonilisse tr�kiladumisse, j��des sealjuures peaaegu
muutumatuks. See sai tuntuks 1960. aastatel Letraset'i lehtede v�ljalaskmisega,
ja hiljuti tekstiredaktoritega nagu Aldus PageMaker, mis sisaldavad erinevaid
Lorem Ipsumi versioone.

\bibliographystyle{plain}
\bibliography{refs}

\end{document}
